\documentclass[parskip=half]{scrreprt}

\usepackage{todonotes} % TODO: remove todonotes package
\newcommand{\todocheck}[1]{\todo[color=blue!40]{#1}}
\usepackage{layouts,layout}


% General
\usepackage[english]{babel}
%\usepackage{placeins} % for \FloatBarrier command

% Links
\usepackage{color,xcolor}
\definecolor{uhd}{RGB}{196,19,47}
\usepackage[colorlinks=true,allcolors=uhd]{hyperref}
\newcommand{\namedeqref}[2]{\hyperref[#2]{#1}~\eqref{#2}}
\newcommand{\appref}[1]{\hyperref[#1]{Appendix~\ref{#1}}}

% Fonts
\usepackage{fontspec,xunicode}
\setmainfont{Palatino}
\setsansfont{Optima}
\setmonofont[Scale=MatchLowercase]{Menlo}

% Symbols
\newcommand{\LCDM}{$\cosmconst$CDM }
\newcommand{\fR}[1]{\(f(\Rscal)\)~#1}

% Equations
\newcommand{\eqpunct}[1]{\mathpunct{#1}}
\newcommand{\eqname}[2]{\label{#2} \\[-1em] \tag*{\llap{\emph{#1}}}}

% Images
\usepackage{pgf}
%\setlength{\abovecaptionskip}{0pt }
\newcommand{\plt}[3]{
	\begin{figure}[ht]
		\begin{center}
			\input{plots/#1.pgf}
		\end{center}
		\caption{\textbf{#2} \quad #3}
		\label{fig:#1}
	\end{figure}
}

% Tables
\usepackage{array} % for math mode in tables
\usepackage{booktabs} % for hline rules

% Page Layout
%\pagestyle{headings}
%\usepackage[textheight=640pt]{geometry}
\usepackage{fancyhdr}
\pagestyle{fancy}
\AtBeginDocument{
	\fancyhf{}
	\lhead{\textit{\nouppercase{\leftmark}}}
	\rhead{\thepage}
	\renewcommand{\headrulewidth}{0.2pt}
}


% Headings
%\usepackage[bf]{titlesec}
%\titleformat{\chapter}[display]{\large\bfseries}{\chaptertitlename\ \thechapter}{5pt}{\large}% NEW
%\titlespacing*{\chapter}{0pt}{30pt}{20pt}% NEW


% Bibliography
\usepackage{natbib}
\bibliographystyle{plainnat}

% Appendix
\usepackage[toc,page]{appendix}

\usepackage{amsmath,amssymb}
\usepackage{esdiff} % derivatives
\usepackage{commath} % math macros
\usepackage{bbm} % blackboard style symbols

% Symbols
\newcommand{\Rscal}{\mathcal{R}} % Ricci Scalar
\newcommand{\Rtens}{R} % Ricci Tensor
\newcommand{\Riemtens}{R} % Riemann Tensor
\newcommand{\Gtens}{G} % Einstein Tensor
\newcommand{\gdet}{|g|} % determinant of g
\newcommand{\Newtconst}{\mathrm{G}_\mathrm{N}} % Newtonian constant
\newcommand{\hcross}{h_\times} % cross-polarized tensor perturbation
\newcommand{\hplus}{h_+} % plus-polarized tensor perturbation
\newcommand{\conft}{\eta} % conformal time
\newcommand{\efold}{\log a} % e-foldings
\newcommand{\Hcosm}{\mathrm{H}} % Hubble function
\newcommand{\Hconf}{\mathcal{H}} % conformal Hubble function
\newcommand{\eosp}{\omega} % equation of state parameter

% Derivatives
\newcommand{\dt}[1]{\diff{#1}{t}}
\newcommand{\ddt}[1]{\diff[2]{#1}{t}}
\newcommand{\dconf}[1]{\dot{#1}}
\newcommand{\ddconf}[1]{\ddot{#1}}
\newcommand{\defold}[1]{#1'}
\newcommand{\ddefold}[1]{#1''}

% Operators and other stuff
\newcommand{\vect}[1]{\boldsymbol{#1}}
\newcommand{\defeq}{:=}
\newcommand{\idmat}{\mathbbm{1}}
\newcommand{\covd}{\nabla}

\usepackage{minted}
\setminted{
	style=xcode,
	linenos,
	autogobble,
	stripnl,
	breaklines,
	escapeinside=||,
	obeytabs=false,
	tabsize=4,
	frame=lines,
	framesep=2.5mm,
	framerule=0.2pt,
}

\newminted{mma}{}
\newmintinline{mma}{}

\newminted{sh}{}
\newmintinline{sh}{}


\AtBeginDocument{
	\title{Gravitational Waves in Modified Gravity}
	% TODO: Subtitle about finding instability in bigravity
	\author{Nils Fischer}
	\date{July 28, 2015}
}


\begin{document}

\begin{titlepage}
\begin{center}
\makeatletter

\Large\textbf{Department of Physics and Astronomy\\
University of Heidelberg}

\vfill

\normalsize
Bachelor Thesis in Physics\\
submitted by\\[0.5cm]
\Large\textbf{\@author}\\[0.2cm]
\normalsize
born in Hannover (Germany)\\[0.5cm]
\Large\textbf{\@date}\\[0.5cm]
\normalsize


\newpage
\thispagestyle{empty}


\LARGE\textbf{\@title}

\vfill

\normalsize
This Bachelor Thesis has been carried out by \textbf{\@author} at the\\
\textbf{Institute for Theoretical Physics} in \textbf{Heidelberg}\\
under the supervision of\\
\textbf{Dr. Valeria Pettorino} \& \textbf{Prof. Dr. Luca Amendola}

\makeatother
\end{center}
\end{titlepage}


\begin{abstract}
	By parametrization of the evolution equation of gravitational waves, I find constraints for the physical viability of a general modified gravity theory. In particular, I focus on a bimetric setting.
\end{abstract}


\tableofcontents


\chapter{Introduction}\label{ch:intro}

%I will first give a brief introduction to standard FRW cosmology in general relativity in \autoref{sec:grav_waves} and derive the evolution equation for gravitational waves by considering perturbations to such a universe. In \autoref{sec:cc_problem}, I will then summarize the cosmological constant problem and explain, why this is still one of the greatest mysteries in theoretical physics of our time. I will briefly explore an approach to solve this problem in a theory of modified gravity. Such modifications will generally also alter the evolution of gravitational waves. I will therefore explain in \autoref{sec:grav_waves_mod} how a modified gravity theory in conjunction with observable data can be tested for physical viability when considering these alterations. \todo{improve}
\todo{write intro}

\todo{c=1}

\section{Gravitational Waves}\label{sec:grav_waves}

%In this section, we will derive the evolution equation of gravitational waves in general relativity. \todo{improve}
\todo{write intro}

\subsection{Gravity and Spacetime in General Relativity}\label{sec:gr}

For a long time, physicists believed space and time were both flat and fundamentally different concepts. In \emph{Newtonian spacetime}, it is therefore straight forward to define notions such as the \emph{length of a curve} and \emph{simultaneity}. \todo{cite \citep{Tolish}}

With Maxwell's theory of electromagnetism and experimental observations regarding the speed of light, however, many of these concepts had to be abandoned. Physicists realized the necessity to reconsider fundamental assumptions about space and time. Albert Einstein addressed many of these issues with his theory of special relativity \todo{summarize SR briefly, mention GR->SR for minkowskian metric} in 1905, but, in particular, one striking observation remained unexplained:

Newtonian mechanics postulates that particles accelerate under the influence of \textbf{any} force proportional to their \emph{inertial mass} \(m_i\). At the same time, the gravitational force a particle induces on another is proportional to its \emph{gravitational mass} \(m_g\) that is completely unrelated to the former in this theory. Strikingly, however, experiments find inertial and gravitational mass indistinguishable from another \todo{order of magnitude?} --- to this end, any two objects will fall with the same velocity, no matter their weight --- and physicists identify both as \emph{mass} \(m=m_i=m_g\) instead. This \emph{equivalence principle} constitutes the basis of Einstein's 1915 theory of \emph{general relativity} where gravity is \textbf{not regarded as an external force} acting on particles in spacetime but rather a phenomenon of the geometry of spacetime itself, with \textbf{particles moving freely in curved spacetime}.

In general relativity, the \emph{Einstein equations}

\begin{align}\label{eq:einstein_eqns}
	\Gtens_{\mu \nu} + \cosmconst g_{\mu\nu} &= 8 \pi \Newtconst \Ttens_{\mu \nu} \\
	\text{with the \emph{Einstein Tensor}} \quad &\Gtens_{\mu \nu} = \Rtens_{\mu \nu} + \frac{1}{2}g_{\mu \nu}\Rscal \eqpunct{,} \label{eq:einstein_tensor} \\
	\text{the \emph{Ricci Scalar}} \quad &\Rscal = g^{\mu \nu}\Rtens_{\mu \nu} \eqpunct{,} \\
	\text{the \emph{Ricci Tensor}} \quad &\Rtens_{\mu \nu} = \Riemtens^\alpha_{\alpha \mu \nu} \eqpunct{,} \\
	\text{the \emph{Riemann Tensor}} \quad &\Riemtens^\alpha_{\beta \mu \nu} = \Gamma^\alpha_{\beta \nu, \mu} - \Gamma^\alpha_{\beta \mu, \nu} + \Gamma^\alpha_{\gamma \mu}\Gamma^\gamma_{\beta \nu} + \Gamma^\alpha_{\gamma \nu}\Gamma^\gamma_{\beta \mu} \\
	\text{and the \emph{Christoffel Symbols}} \quad &\Gamma^\alpha_{\beta \gamma} = \frac{1}{2} g^{\alpha \mu} \del{g_{\mu \beta, \gamma} + g_{\mu \gamma, \beta} - g_{\beta \gamma, \mu}}
\end{align}
relate the geometry of spacetime, encoded in the metric tensor~\(g_{\mu \nu}\), to the energy content of the universe~\(\Ttens_{\mu \nu}\). The \emph{cosmological constant} \(\cosmconst\) is a free parameter in this theory and is discussed in detail in \autoref{sec:cc_intro}.

We can obtain the Einstein equations through variation of the \emph{Einstein-Hilbert action}
\begin{align}\label{eq:einstein_hilbert_action}
	S[g] = \int \! \dif{^4x} \sqrt{-\metrdet{g}} \del{\Rscal - 2\cosmconst} + \int \! \dif{^4x} \sqrt{-\metrdet{g}} \lagrdensmatter \quad \text{with} \quad \metrdet{g} = \det{g}
\end{align}
with respect to \(g_{\mu \nu}\). A detailed derivation is done in \appref{app:deriv_einstein_eqns}. The matter Lagrangian denoted by \(\lagrdensmatter\) is only \emph{minimally coupled} to gravity through the measure \(\sqrt{-\metrdet{g}}\dif{^4x}\) that arises from the spacetime metric in general relativity.

Given a stress-energy tensor \(\Ttens_{\mu \nu}\), the Einstein equations constitute ten \todo{why ten? > symmetric} highly non-linear partial differential equations for the spacetime metric~\(g\). The metric, in turn, restores the notion of the length of a curve in spacetime and thus allows us to formulate postulates for the dynamics of matter in the universe.

In particular, as a generalization of Newton's law, particles without any external forces acted upon will move on \emph{geodesics} in spacetime, i.e. their curve is stationary with respect to the length functional.\todo{explain better}

A simple solution to the Einstein equations \todo{in vacuum?} is the flat \emph{Minkowskian} spacetime metric
\begin{equation}
	g_{\mu \nu} = \eta_{\mu \nu} =
	\begin{bmatrix}
		-1 & 0 \\
		0 & \idmat_3
	\end{bmatrix}_{\mu \nu}
\end{equation}
of special relativity.\todo{include this here, or elaborate?}


\subsection{FRW Cosmology}\label{sec:frw}

In cosmology, we strive to understand \textbf{how the entire universe evolves}.\todo{elaborate}

As of the \emph{cosmological principle}, it is reasonable to assume the universe is \emph{spatially homogeneous and isotropic} at large scales. This gives rise to six spatial symmetries \todo{include derivation with Killing vector fields?} and leads to the \emph{Friedmann-Robertson-Walker~(FRW)} metric
\begin{align}
	\dif{s}^2 = -\dif{t}^2 + a(t)^2 \gamma_{ij} \dif{x^i}\dif{x^j} \quad \text{where} \quad &\dif{s}^2 = g_{\mu \nu}\dif{x}^\mu \dif{x}^\nu \\
	\text{and} \quad &\gamma_{ij}(r,\theta,\phi) =
	\begin{bmatrix}
		\frac{1}{1-\spatcurv r^2} & 0 & 0 \\
		0 & r^2 & 0 \\
		0 & 0 & r^2 sin^2(\theta)
	\end{bmatrix}_{ij}
\end{align}
that is a particular solution to the Einstein equations describing a smooth, expanding universe. The \emph{scale factor}~\(a(t)\) is the only freedom left after considering the symmetries and scales the time-independent metric \(\gamma_{ij}\) of a spatial \todocheck{can the subspace(?) be called spatial, or rather three-dim.?} subspace with constant curvature~\(\spatcurv\). See \appref{app:deriv_frw} for a detailed derivation.

It is important to note that distances described by the FRW coordinates are merely \emph{coordinate distances} or \emph{comoving distances} that remain constant even in an expanding universe. Comoving distances are scaled by the time-dependent scale factor \(a(t)\) to obtain the physical distances. Similarly, one can define the \emph{comoving} or \emph{conformal time}
\begin{equation}
	\conft = \int_0^t \frac{\dif{t^\prime}}{a(t^\prime)}
\end{equation}
as the comoving distance light (with \(\dif{s}^2=0\)) could have traveled since \(t=0\). The conformal time thus defines the causal structure in comoving coordinates and is also called the \emph{comoving horizon}. It is often convenient to parametrize the evolution of the universe in conformal time \(\conft\) instead of cosmic time \(t\). Another parametrization we will use are \emph{e-foldings} \(\efold\). For the remainder of this thesis, derivatives by conformal time and e-foldings will be denoted by dots and primes, respectively, as in
\begin{equation}
	\diff{}{\conft}\equiv\dconf{} \quad \text{and} \quad \diff{}{\efold}\equiv\defold{}
	\eqpunct{.}
\end{equation}

Given the particular form of the spacetime metric, the \namedeqref{Einstein tensor}{eq:einstein_tensor} can be explicitly computed. This is done in \appref{app:deriv_frw_einstein}. When we also assume the universe is filled with matter that, at large scales, resembles a perfect fluid \todo{add reference} with stress-energy tensor
\begin{equation}\label{eq:perfect_fluid}
	\Ttens^{\mu\nu} = \del{\rho + p}u^\mu u^\nu + p g^{\mu\nu} \quad \text{in time direction} \quad u^\mu = \del{1,0,0,0}^T
\end{equation}
with \emph{matter density} \(\dens\), \emph{pressure} \(p\) and linear \emph{equation of state}
\begin{equation}
	p = \eosp \rho \eqpunct{,}	
\end{equation}
the \namedeqref{Einstein equations}{eq:einstein_eqns} reduce to two ordinary, coupled differential equations \todo{add reference} \todo{check lambda dependence}
\begin{subequations}\label{eq:friedmann_eqns}
\begin{align}
	\ddt{a} &= -\frac{4\pi}{3}\Newtconst\del{\rho + 3p}a + \frac{\cosmconst}{3} \eqname{acceleration equation}{eq:acceleration} \\
	\Hcosm^2 &= \frac{8\pi}{3}\Newtconst\rho - \frac{\spatcurv}{a^2} + \frac{\cosmconst}{3} \eqname{Friedmann equation}{eq:friedmann}
\end{align}
\end{subequations}
with the \emph{Hubble function}
\begin{equation}\label{eq:hubble}
	\Hcosm \equiv \dt{a}\frac{1}{a} \quad \text{or} \quad \Hconf \equiv \frac{\dconf{a}}{a} = a \Hcosm \quad \text{in conformal time \(\conft\).}
\end{equation}
The \emph{equation of state parameter} \(\eosp\) for various matter types is summarized in \autoref{tab:matter_types}. For a universe filled with any such matter, we can solve the \namedeqref{Friedmann equations}{eq:friedmann_eqns} to find both the time evolution of the matter density
\begin{equation}\label{eq:frw_dens_evol}
	\dens \propto a^{-\nexp}
\end{equation}
and of the scale factor
\begin{equation}\label{eq:frw_a_evol}
	a(t) \propto
	\begin{cases}
		t^{\frac{2}{\nexp}} &\text{for \(\eosp\neq-1\)} \\
		\eul^{\Hcosm t} &\text{for \(\eosp=-1\)}
	\end{cases} \quad \text{with} \quad \nexp \defeq 3\del{1 + \eosp} \eqpunct{.}
\end{equation}
This result is derived in \appref{app:deriv_frw_a_evol} and already exhibits a wealth of cosmological implications. In a universe filled with matter of \(\eosp\neq-1\) we find \(\Hcosm \propto \frac{1}{t}\), for example, and can therefore identify it with a notion of the \emph{age of the universe}. Furthermore, the relation \(\dens \propto t^{-2}\) that follows from \eqref{eq:frw_dens_evol} in such a universe implies a singularity at \(t=0\) where the density becomes infinite. This is called the \emph{big bang}.

\begin{table}[ht]
\centering
\begin{tabular}{l>{$}c<{$}>{$}c<{$}>{$}c<{$}}
	\toprule
	~ & i & \eosp & \nexp \\
	\midrule
	radiation & \symbrad & \frac{1}{3} & 4 \\
	dust & \symbdust & 0 & 3 \\
	cosmological constant & \cosmconst & -1 & 0 \\
	spatial curvature & \spatcurv & -\frac{1}{3} & -2 \\
	\bottomrule
\end{tabular}
\caption{Overview of cosmological properties for a universe filled with various matter types}
\label{tab:matter_types}
\end{table}

Multiple matter types in the universe combine to the total matter density
\begin{equation}
	\rho(t) = \sum_i \rho_i(t) \quad \text{with equation of state} \quad p_i = \eosp_i\rho_i \quad \text{each,}
\end{equation}
where we can include the effect of a spatial curvature \(\spatcurv\) and a cosmological constant \(\cosmconst\) through the definition of additional pseudo-densities
\begin{equation}
	\dens_\spatcurv(t) = -\frac{3}{8\pi\Newtconst}\frac{\spatcurv}{a^2} \quad \text{and} \quad \dens_\cosmconst(t) = \frac{\cosmconst}{8\pi\Newtconst}.
\end{equation}
When we then define the relative matter densities \todo{explain}
\begin{equation}
	\Dens_i(t) \defeq \frac{\rho_i(t)}{\denscrit} \quad \text{with} \quad \denscrit \defeq \frac{3\Hcosm_0^2}{8\pi\Newtconst}
\end{equation}
where \(\Hcosm_0\) denotes the value of the Hubble function today at cosmic time \(t=t_0\), the \namedeqref{Friedmann equation}{eq:friedmann} becomes
\begin{equation}\label{eq:friedmann_normalized_dens}
	\frac{\Hcosm^2}{\Hcosm_0^2} = \sum_i \Dens_i(t)
\end{equation}
and we find from \eqref{eq:frw_dens_evol} the relation
\begin{equation}
	\Dens_\cosmconst \propto a^2 \Dens_\spatcurv \propto a^3 \Dens_\symbdust \propto a^4 \Dens_\symbrad \eqpunct{.}
\end{equation}
Because the universe expands monotonically with time for any of these matter types, as given by \eqref{eq:frw_a_evol}, this result allows us to consider successive \emph{cosmological regimes} in an FRW universe with a dominant matter type each. Radiation dominates in the early universe and is followed by a regime of dust domination (also called \emph{matter domination}). The spatial curvature \(\spatcurv\) is measured to be zero very precisely today, thus eliminating the corresponding regime. Today, the universe is in a regime dominated by a cosmological constant (also called \emph{de Sitter space}) that is discussed in more detail in \autoref{sec:cc_intro}.


\subsection{Tensor Perturbations in an FRW Universe}\label{sec:perturb}

At smaller scales, the universe is not homogeneous and isotropic at all, of course. Galaxies, stars and planets, as well as radiation or, in fact, any energy content of the universe disturb the spacetime metric locally\todo{also primordial / early universe / inflation and such}\todo{..and lead to gravitational interaction}. It is therefore reasonable to consider perturbations~\(\delta g\) around the smooth FRW metric and their evolution.\todo{motivate better > baccigalupi p. 22}

When we assume an exact solution~\(g\) to the unperturbed Einstein equations and consider a sufficiently small perturbation to the stress-energy tensor~\(\delta \Ttens_{\mu \nu}\), then the metric perturbation \(\delta g\) that solves
\begin{equation}
	\Gtens_{\mu \nu}[g + \delta g] = \Ttens_{\mu \nu}[g] + \delta \Ttens_{\mu \nu}[g] \quad \text{for} \quad 8 \pi \Newtconst = 1
\end{equation}
will also be small \todo{really?} and we obtain
\begin{equation}\label{eq:perturbed_einstein_eqns}
	\delta\Gtens_{\mu \nu}[g,\delta g] = \delta \Ttens_{\mu \nu}[g]
\end{equation}
with \(\delta\Gtens_{\mu \nu}[g,\delta g]\) linear in \(\delta g\) in linear perturbation theory.

Because of its symmetry condition\todo{add reference or derive}, \(\delta g\) has 10 degrees of freedom that we can parametrize as \citep{Schulz}
\begin{align}
	\delta g = -2A \dif{x^0}\otimes\dif{x^0} + B_i\del{\dif{x^0}\otimes\dif{x^i}+\dif{x^i}\otimes\dif{x^0}} + \del{2C\gamma_{ij} + 2E_{ij}}\dif{x^i}\otimes\dif{x^j}
\end{align}\todocheck{notation}
for small spatial \emph{scalar fields}~\(A=A(x^0)\) and~\(C=C(x^0)\), a \emph{vector field}~\(B_i=B_i(x^0)\) and a symmetric, trace-free \emph{tensor field}~\(E_{ij}=E_{ij}(x^0)\).

As of the \emph{Helmholtz theorem}, the parameters uniquely decompose further into scalar, vector and tensor components~as
\begin{align}
	\delta g = \delta g^{\mathrm{scalar}} + \delta g^{\mathrm{vector}} + \delta g^{\mathrm{tensor}} \eqpunct{.}
\end{align}
This is shown in detail in \appref{app:deriv_decompose} and allows us to study scalar, vector and tensor perturbations separately. \todo{add info about scalar, vector, tensor > grav. waves}

\emph{Gravitational waves} now arise when we only consider \textbf{unsourced tensor perturbations} of the metric. This is analogous to the propagation of electromagnetic waves in vacuum, for example. \todo{elaborate} In fact, scalar and vector perturbations can only arise from stress-energy perturbation. \todo{add reference} Therefore, only tensor perturbations remain for \(\delta \Ttens_{\mu \nu}=0\).

It is shown in \appref{app:deriv_decompose} that
\begin{equation}
	\delta g^{\mathrm{tensor}}_{ij} = h_{ij}
\end{equation}
is a \emph{symmetric, traceless, divergence-free tensor field} and can therefore be expressed in terms of two functions~\(\hcross\) and~\(\hplus\)~as
\begin{equation}
	h_{ij} =
	\begin{bmatrix}
		\hplus & \hcross & 0 \\
		\hcross & -\hplus & 0 \\
		0 & 0 & 0
	\end{bmatrix}_{ij}
\end{equation}
with an implicit choice of the \(z\)-axis in direction of the \emph{wave vector}~\(\vect{k}\) \citep{Dodelson}. \todo{elaborate choice of axis, show div/trace-free with wave-vector k, "transverse-traceless TT gauge"}

The perturbed line element becomes
\begin{equation}
	\dif{s}^2 = -\dif{t}^2 + a(t)^2 \del{\gamma_{ij} + h_{ij}} \dif{x^i}\dif{x^j}
\end{equation}
and with this explicit form of the metric we can compute the Einstein tensor perturbation in \eqref{eq:perturbed_einstein_eqns}. This is done in \appref{app:deriv_gtens_perturb} and we obtain
\begin{align}\label{eq:evolution}
	\delta\Gtens_{ij} = \delta\Rtens_{ij} = \frac{3 a^2}{2} \Hcosm \dt{h_{ij}} + \frac{a^2}{2}\ddt{h_{ij}} + \frac{k^2}{2}h_{ij} \eqpunct{.}
\end{align}

The \namedeqref{perturbed Einstein equations}{eq:perturbed_einstein_eqns} that govern the evolution of the metric perturbations then become a wave equation \todo{damped oscillator, not wave!?}
\begin{subequations}\label{eq:grav_waves_evolution}
\begin{align}
	\ddt{h} + 3 \Hcosm \dt{h} + \frac{k^2}{a^2} h &= 0 \quad \text{in cosmic time \(t\)} \\
	\text{or} \quad \ddconf{h} + 2 \Hconf \dconf{h} + k^2 h &= 0 \quad \text{in conformal time \(\conft\)} \label{eq:grav_waves_evolution_conft}
\end{align}
\end{subequations}
for \(h \in \cbr{\hcross, \hplus}\). Its solutions are called \emph{gravitational waves} and occur in two independent \emph{polarizations} \(\hcross\) and \(\hplus\). Neglecting the friction term, harmonic oscillations
\begin{equation}
	h \propto \eul^{\pm k \conft} \quad \text{with the \emph{wavelength mode}} \quad k \equiv \abs{\vect{k}}
\end{equation}
solve \eqref{eq:grav_waves_evolution_conft}.

Gravitational waves are damped by the expansion of the universe as exhibited by \eqref{eq:grav_waves_evolution} where the friction term is proportional to the Hubble function. In an expanding universe, the \namedeqref{Hubble function}{eq:hubble} is positive, thus damping the amplitude of gravitational waves. However, the tensor perturbations will remain constant at early times where the conformal time is still smaller than the wavelength scal~\(\frac{1}{k}\) of the gravitational wave. It begins to oscillate at its \emph{horizon entry}~\(\conft \simeq \frac{1}{k}\) where cosmological scales on the order of the gravitational wave's wavelength move into causal contact. The horizon entry for larger-scale modes with smaller \(k\) thus occurs later. This behaviour is presented in \autoref{fig:varying_k} where the absolute value~\(\abs{h(a)}\) of several tensor modes in a universe dominated by the cosmological regimes discussed in \autoref{sec:frw} is plotted in double-logarithmic scale. Detailed information about the assumptions, initial conditions and parameters that were chosen to obtain the plot in \autoref{fig:varying_k} are given in the context of a parametric modification to the \namedeqref{evolution equation of gravitational waves}{eq:grav_waves_evolution} in \autoref{sec:param_friction_const}.

\plt{varying_k}{Tensor perturbations \(\abs{h(a)}\) for different wavelength modes \(k\)}{In an expanding universe, tensor perturbations remain constant at early times until cosmological scales on the order of the gravitational wave's wavelength scale \(\frac{1}{k}\) move into causal contact. This horizon entry occurs later for large-scale gravitational waves with smaller modes \(k\).}


\todo{observations/CMB, limits}


We will further discuss gravitational waves in the context of \emph{modified gravity} in \autoref{ch:param_mod_grav} and the remainder of the thesis. However, to understand the reason why a modification of general relativity may be necessary, \autoref{sec:cc_problem} first formulates the \emph{cosmological constant problem} and an approach for solving it.

\section{The Cosmological Constant Problem}\label{sec:cc_problem}

Both quantum field theory and general relativity are extremely well-tested theories and constitute the basis of modern physics, both in their respective fields. Whereas quantum field theory succeeds remarkably well in predicting particle physics phenomena, general relativity celebrates an equal success in large-scale cosmological observations. Both theories and particularly their interplay are not without mysteries\todo{better: not fully understood?}, however.

This section strives to formulate the \emph{cosmological constant problem} and particularly emphasizes its problem of \emph{radiative instability}. The quest for solutions will lead us to consider modified theories of gravity that will be the focus for the remainder of this thesis.
%General relativity includes a free parameter \(\cosmconst\), called \emph{cosmological constant}, that will be introduced in \autoref{sec:cc_intro}. One mystery of quantum field theory, however, is the concept of \emph{vacuum energy} discussed in \autoref{sec:cc_contrib}. Interestingly enough, both exhibit a very similar interpretation and consolidating the two seems a very natural approach. The problem is more complex, though, and connects unsolved phenomena of both quantum field theory and general relativity. This section strives to formulate the \emph{cosmological constant problem} and particularly emphasizes its problem of \emph{radiative instability} in \autoref{sec:rad_instability}. \autoref{sec:new_cc_problem} finally suggests a modern approach to this century-old problem where we already assume a solution for part of it. This leads to the suggestion of a \emph{modified theory of gravity} to solve the remaining problem of the accelerated expansion of our universe.

\subsection{The Cosmological Constant}\label{sec:cc_intro}

\emph{Lovelock's theorem}\label{sec:lovelock} states that general relativity as it emerges from the Einstein-Hilbert action \ref{eq:einstein_hilbert_action} is, in fact, the \textbf{unique} metric theory in four spacetime dimensions that \todo{add reference}
\begin{itemize}
	\item gives rise to second-order equations of motion
	\item for only one symmetric rank-2 tensor
	\item that is local and lorentz-invariant.\todo{Bianchi-identities?}
\end{itemize}
This includes a free parameter \(\cosmconst\) of the theory that is called the \emph{cosmological constant}. Its value does not follow from the theory and thus it can only be constrained experimentally.

It appears in the Einstein equations \ref{eq:einstein_eqns} as a contribution
\begin{equation}
	\Ttens_{\cosmconst\textnormal{,}\mu\nu} = - \cosmconst g_{\mu\nu}
\end{equation}
to the stress-energy tensor that corresponds to a \textbf{homogeneous energy density} penetrating the entire spacetime. In standard FRW cosmology, such a contribution results in an accelerating expansion of our universe as derived in detail in \appref{app:deriv_acc_exp_lambda}.

In fact, cosmological observations clearly suggest such an accelerated expansion taking place at recent cosmic times. \todo{add references} The \LCDM (\emph{\(\cosmconst\) cold dark matter}) standard model of cosmology therefore includes a cosmological constant on the order of \(10^{-122}\) in Planck units and agrees with all \todo{really?} cosmological observations remarkably well. In this theory, the physical origin of the homogeneous energy contribution that the cosmological constant represents remains a mystery, however, and is given the elusive \todo{good word?} name \emph{dark energy}. \todo{add reference}

The fact that the cosmological constant must be fixed by observations is not remarkable on its own, of course, since also the gravitational constant \(\Newtconst\) must be determined experimentally. Theories such as the standard model of particle physics include a number of such free parameters. This poses an entirely different, somewhat philosophical problem that is often naively answered using anthropic arguments. This is discussed in detail in \appref{app:anthropic}.

\subsection{Contributions to the Cosmological Constant}\label{sec:cc_contrib}

The cosmological constant problem arises when we consider both classical and quantum phenomena that, to our knowledge, should contribute to the cosmological constant.

Already in basic quantum mechanics, the uncertainty principle requires every physical system to have a zero-point energy\todo{reference}. This immediately carries over to quantum field theory, where a (free) field is an infinite collection of coupled quantum mechanical harmonic oscillators. With a zero-point energy each, they combine to an infinite \emph{vacuum energy} \(\rho_\textnormal{vac}\). \todo{add some derivation?}
 
In quantum field theory, the vacuum energy is largely ignored as only differences in energy determine the dynamics of the system. In general relativity, however, all energy content gravitates, including the vacuum energy. \todo{reference} \todo{how about Lamb shift, Casimir effect?}

Remarkably, assigning an importance to not only energy differences, but also absolute energy values gives rise to another, entirely classical contribution to the stress-energy tensor, namely the \emph{zero-point potential} \(V_0\). This is the energy
\begin{equation}
	\Ttens_{\mu\nu} = - V_0 g_{\mu\nu}
\end{equation}
where the kinetic energy vanishes and the potential assumes its minimum value \(V_0\). This zero-point potential is usually chosen arbitrarily when only energy differences are considered but must be taken into account in general relativity. Particularly in presence of phase transitions, one can generally not choose the zero-point potential such that it always vanishes.
%When we simply consider a scalar field \(\Phi\) with action
%\begin{equation}
%	S_\Phi = \int \! \dif{^4x} \sqrt{-\metrdet{g}} \del{\frac{1}{2}\partial_\mu\Phi\partial^\mu\Phi - V(\Phi)}
%\end{equation}
%and corresponding stress-energy tensor
%\begin{equation}
%	\Ttens_{\Phi\textnormal{,}\mu\nu} = \partial_\mu\Phi\partial_\nu\Phi - g_{\mu\nu}\del{\frac{1}{2}\partial_\sigma\Phi\partial^\sigma\Phi + V(\Phi)}
%\end{equation}
%it becomes obvious that the minimum energy is reached for
%\begin{equation}
%	\Ttens_{\Phi_\textnormal{min}\textnormal{,}\mu\nu} = - V_0 g_{\mu\nu} \quad \text{with} \quad V_0 \defeq V(\Phi_\textnormal{min})
%\end{equation}
%where the kinetic energy vanishes and the potential assumes its minimum value \(V_0\). This zero-point potential is usually chosen arbitrarily when only energy differences are considered but must be taken into account in general relativity. Particularly in presence of phase transitions, one can generally not choose the zero-point potential such that it always vanishes.

The free parameter \(\cosmconst\) in the Einstein equations therefore combines with both the quantum vacuum energy \(\rho_\textnormal{vac}\) and the classical zero-point potential \(V_0\) of every quantum field in the universe to an \emph{effective} cosmological constant
\begin{equation}
	\cosmconst_\textnormal{eff} = \cosmconst + \rho_\textnormal{vac} + V_0
\end{equation}
that we measure as dark energy. In comparison to the small value for \(\cosmconst_\textnormal{eff}\) we observe today, however, both contributions are extremely large \citep{Martin2012}. \todo{add reference} This already suggests a severe \emph{fine-tuning} problem where the value of the original cosmological constant \(\cosmconst\) must precisely cancel the other contributions up to the small value we measure today. \todo{elaborate fine-tuning} This is not the entire cosmological constant problem yet, though.

\subsection{Radiative Instability}\label{sec:rad_instability}

The full scope of the problem arises when we consider in more detail the vacuum energy that we found to be infinite before. The mechanism to make sense of divergencies like this in quantum field theory is the framework of \emph{renormalization}. In the process to find a find finite, \emph{renormalized} value for \(\rho_\textnormal{vac}\) one generally adds counterterms for every order in perturbation theory that each depend on an \emph{arbitrary subtraction scale}.

Generally, successive orders are not significantly suppressed by a sufficiently small perturbation parameter \(\cosmconst\), however. In fact, for the standard model Higgs field the self-coupling parameter of perturbation \(\cosmconst\) is of the order \(10^{-1}\) \todo{add reference} and therefore every order in perturbation theory must be renormalized independently. This \emph{radiative instability} requires us to fine-tune the cosmological constant \textbf{repeatedly} for every order in perturbation theory and thus makes it sensitive even to small-scale physics where we assume our theory to break down \citep{Datta1996}.

This also prohibits us from finding an \emph{effective theory} where the full structure of perturbation theory is encoded in one finite, renormalized value by means of a \emph{Wilson effective action}. \todo{add reference} Again, we find the renormalized vacuum energy is unstable against changes in the unknown UV-regime of the theory \citep{Datta1996}.

\todo{how about \emph{why now?}}
\todo{how about naturalness <-> anthropic \cite{Datta1996}?}
\todo{ask Wetterich > fixed points}

\subsection{The \emph{New} Cosmological Constant Problem and Modified Gravity}\label{sec:cc_problem_new}

The cosmological constant problem is deeply rooted in our inability to find a renormalized vacuum energy that is stable against changes in its effective description. Therefore, an approach to the problem is to assume that some mechanism makes the vacuum energy vanish altogether instead and then find another theory that explains the non-zero cosmological constant we observe today.

In fact, unbroken \emph{supersymmetry} would accomplish just that. In supersymmetry, bosons and fermions are related by a symmetry and share the same mass. Supersymmetric partners contribute to the vacuum energy with opposite signs, however, thus precisely canceling each other.

With the cosmological constant set to zero, the cosmological observation of accelerated expansion \todo{other observables?} remains to be explained by a different mechanism. Dark energy models such as the \emph{quintessence} theory \todo{add reference} postulate further contributions to the stress-energy tensor that have the same accelerating effect as a cosmological constant. A different approach is to modify the theory of gravity instead.

Every attempt towards a modified theory of gravity has to take Lovelock's theorem into consideration and break at least one of its assumptions that make general relativity unique. The concept of the entire class of \emph{\fR{theories}}, for example, is to replace the Ricci scalar \(\Rscal\) in the \namedeqref{Einstein-Hilbert action}{eq:einstein_hilbert_action} by a function \(f(\Rscal)\).
% such that the action becomes
%\begin{equation}\label{eq:fR_action}
%	S[g] = \int \! \dif{^4x} \sqrt{-\metrdet{g}} f(\Rscal) + \int \! \dif{^4x} \sqrt{-\metrdet{g}} \lagrdensmatter
%	\eqpunct{.}
%\end{equation}
\fR{theories} break Lovelock's assumption of only second-order equations of motion and are in many cases equivalent to an additional scalar field contribution to the matter Lagrangian that is non-minimally coupled to gravity. \todo{add reference}

Alternatively, the hypothesis of a \emph{massive graviton} requires a second, arbitrary \emph{reference metric} \(f\) in addition to the physical metric \(g\). The additional metric is necessary to construct a mass term because a single metric allows only trivial self-interaction terms \(g^{\mu\sigma}g_{\sigma\nu}=\delta^\mu_\nu\) and \(g^{\mu\nu}g_{\mu\nu} = 4\). By postulating a symmetric action where the reference metric \(f\) behaves dynamically just like \(g\), we arrive at the theory of \emph{massive bigravity}.
% with the action
%\begin{equation}
%	S[g,f] =
%	- \frac{\planckM_g^2}{2} \int \! \dif{^4x} \sqrt{-\metrdet{g}} \Rscal[g]
%	- \frac{\planckM_f^2}{2} \int \! \dif{^4x} \sqrt{-\metrdet{f}} \Rscal[f]
%	+ m^2 \planckM_g^2 \int \! \dif{^4x} \sqrt{-\metrdet{g}} \sum_{n=0}^4 \beta_n e_n()
%	+ \int \! \dif{^4x} \sqrt{-\metrdet{g}} \lagrdensmatter
%\end{equation}
The cosmology in this theory allows solutions with late-time acceleration without a cosmological constant. \todo{add reference}

\todo{connect to next chapter}



\chapter{Parametrization of Modified Gravitational Wave Evolution}\label{ch:param_mod_grav}

Many theories of modified gravity are discussed in the literature \todo{add reference} and this thesis will not focus on one specific model. Instead, the remainder of the thesis will explore various parametric modifications to the \namedeqref{evolution equation of gravitational waves}{eq:grav_waves_evolution} that can result from a modified gravity theory. For such a modified gravity theory to be physically viable, the metric tensor perturbations must remain within constraints set by observations. In particular, any theory that exhibits \textbf{growing} tensor modes in cosmological evolution should be regarded with serious doubt, as their amplitude would likely be large enough today to be excluded by experiments. \todo{mention observations} \todo{add reference}

In an \fR{theory} without anisotropic stress, for example, the evolution equation becomes \citep{Xu2015,Hwang1996}
\begin{equation}
	\ddconf{h} + \del{2 + \diff{\log{F}}{\efold}} \Hconf \dconf{h} + k^2 h = 0 \quad \text{with} \quad F \defeq \diff{f(\Rscal)}{\Rscal} \eqpunct{,}
\end{equation}
thus adding an additional friction term. Furthermore, the propagation speed of gravitational waves in a modified gravity theory can deviate from the speed of light \(c\) \citep{Amendola2014,Raveri2014}. I investigate the effects such modifications have on the evolution of gravitational waves by introducing appropriate parameters in \eqref{eq:grav_waves_evolution}. A reasonable parametric modification motivated by the above considerations is
\begin{subequations}\label{eq:evolution_friction}
\begin{align}
	\ddt{h} + \del{3 + \alphaM} \Hcosm \dt{h} + \frac{\cT^2 k^2}{a^2} h &= 0 \quad \text{in cosmic time \(t\),} \\
	\ddconf{h} + \del{2 + \alphaM} \Hconf \dconf{h} + \cT^2 k^2 h &= 0 \quad \text{in conformal time \(\conft\)} \\
    \text{or} \quad \ddefold{h} + \del{\frac{\defold{\Hconf}}{\Hconf} + 2 + \alphaM} \defold{h} + \frac{\cT^2 k^2}{\Hconf^2} h &= 0 \quad \text{in e-foldings \(\efold\)}
\end{align}
\end{subequations}
where \(\alphaM\) denotes an additional friction term and \(\cT\) a deviation of the propagation speed from the speed of light. Both parameters may in general be time-dependent. The standard behaviour of general relativity discussed in \autoref{sec:perturb} is recovered for \(\alphaM=0\) and \(\cT=1\).


\section{Constant additional friction}\label{sec:param_friction_const}

I will first consider a constant friction term \(\alphaM=\const\) in the \namedeqref{parametrized evolution equation}{eq:evolution_friction}. \autoref{fig:varying_aM} shows the numerical solution of the equation for various values of \(\alphaM\). The result was obtained with the \mmainline{ParametricNDSolve} function in Mathematica where several assumptions were made:

First, initial conditions were chosen such that
\begin{equation}
	h(a \to 0) = 1 \quad \text{and} \quad \defold{h}(a \to 0) = 0 \eqpunct{.}
\end{equation}\todo{write the limit like this?}
Furthermore, a standard FRW background was assumed where the Hubble function is given by the \namedeqref{Friedmann equation}{eq:friedmann_normalized_dens}
\begin{equation}
	\Hcosm(t) = \Hcosm_0 \sqrt{\sum_i\Dens_i(t)}
	\eqpunct{.}
\end{equation}
Only matter and dark energy contributions \(\Dens_\symbdust\) and \(\Dens_\cosmconst\) were considered here because they dominate in late-time cosmological regimes as discussed in \autoref{sec:frw}. Values for \(\Hcosm_0\) and \(\Dens_\symbdust(t_0)\) in
\begin{equation}
	\Dens_\symbdust(t) = \Dens_\symbdust(t_0) a^{-3} \quad \text{and} \quad \Dens_\cosmconst(t) = 1 - \Dens_\symbdust(t_0)
\end{equation}
were obtained from the \cite{Planck} \todo{add reference to planck} results given in \autoref{tab:planck_data}.
Lastly, the physical amplitude for the gravitational waves
\begin{equation}
	h_\textnormal{phys}(a) = \sqrt{A_\tenspert\del{\frac{k}{k_0}}^{n_\tenspert}} \cdot h(a)
\end{equation}
was also computed from the \cite{Planck} \todo{add reference} results given in \autoref{tab:planck_data}. \todo{explain phys. ampl.!}

\plt{varying_aM}{Tensor perturbations \(\abs{h(a)}\) with additional friction \(\alphaM=\const\)}{Increasing \(\alphaM\) introduces more friction, so that the amplitude of the gravitational wave decreases more rapidly, but also delays the horizon entry.}

In \autoref{fig:varying_aM}, the absolute value \(\abs{h(a)}\) is plotted in double-logarithmic scale \todo{better word?} for both positive and negative values of \(\alphaM\). Clearly, the amplitude of the gravitational wave decreases more rapidly for larger \(\alphaM\) because additional friction is introduced. However, also the horizon entry is delayed for larger \(\alphaM\) thus introducing a competing effect. This reproduces the behaviour found in \cite{Pettorino2014}. \todo{write this here?}


\subsection{Analytic Solution in Cosmological Regimes}

The \namedeqref{parametrized evolution equation}{eq:evolution_friction} can even be solved analytically in cosmological regimes dominated by one of the matter types discussed in \autoref{sec:frw}. The expansion of an FRW universe dominated by matter or radiation is given by \eqref{eq:frw_a_evol} or
\begin{equation}
	a(\conft) \propto \conft^{\nexpconft} \quad \text{with} \quad \nexpconft \defeq \frac{2}{1+3\eosp} \quad \text{in conformal time.}
\end{equation}
This allows us to find an expression \(\Hconf = \frac{\dconf{a}}{a} = \frac{\nexpconft}{\conft}\) for the Hubble function so that \eqref{eq:evolution_friction} becomes
\begin{equation}\label{eq:evolution_friction_regime}
	\ddconf{h} + \del{2 + \alphaM} \frac{\nexpconft}{\conft} \dconf{h} + \cT^2 k^2 h = 0
	\eqpunct{.}
\end{equation}
This is a Bessel differential equation that has solutions in terms of Bessel functions as discussed in \appref{app:bessel}. Thus, \eqref{eq:evolution_friction_regime} is solved by
\begin{equation}
	h(\conft) = \conft^{-p} \sbr{C_1 \besselJ_p(\cT k \conft) + C_2 \besselY_p(\cT k \conft)} \quad \text{with} \quad p = \nexpconft\del{1 + \frac{\alphaM}{2}}-\frac{1}{2}
\end{equation}
where \(\besselJ_p(x)\) and \(\besselY_p(x)\) denote the Bessel functions of first and second kind, respectively. Their asymptotic behaviour for large \(k\) or late times \(\conft\) is given in \appref{app:bessel} and both correspond to oscillations with an amplitude decreasing as \(\conft^{-\frac{1}{2}}\). Therefore, the amplitude of gravitational waves with additional friction \(\alphaM\) behaves as
\begin{subequations}\label{eq:slope_const_friction}
\begin{align}
	h(\conft) &\propto \conft^{-p-\frac{1}{2}} = \conft^{-\nexpconft\del{1+\frac{\alphaM}{2}}} \\
	\text{or} \quad h(a) &\propto a^{-\frac{1}{\nexpconft}\del{p-\frac{1}{2}}} = a^{-\del{1+\frac{\alphaM}{2}}} \label{eq:slope_const_friction_a}
\end{align}
\end{subequations}
times fast oscillation in this regime.

This result immediately gives a constraint for the \(\alphaM\) parameter such that growing tensor modes are avoided. Since \eqref{eq:slope_const_friction_a} exhibits a growing amplitude of \(h(a)\) for a positive exponent \(-\del{1+\frac{\alphaM}{2}} > 0\) in cosmological expansion, any theory of gravity that modifies the \namedeqref{evolution equation of gravitational waves}{eq:evolution_friction} by an additional friction term \(\alphaM=\const\) should fulfill
\begin{equation}\label{eq:grow_cond_const_fric}
	\alphaM \geq -2
\end{equation}
or else be regarded with serious doubt. \todo{"in tension with exp."?}

The numerical result obtained before reflects this behaviour as presented in \autoref{fig:growing_aM}. The plot shows the evolution of tensor modes for both \(\alphaM = -2\) and \(\alphaM < -2\) using the same numerical solution as in \autoref{fig:varying_aM} as well as the slope of the amplitude obtained analytically in \eqref{eq:slope_const_friction}. As expected from the analytic considerations, the gravitational wave with \(\alphaM = -2\) remains stable whereas the gravitational wave with \(\alphaM < -2\) grows in amplitude.

\plt{growing_aM}{Tensor perturbations \(\abs{h(a)}\) with additional friction \(\alphaM \leq -2\)}{The numerical solution (solid lines) agrees with the slope of the amplitude obtained analytically (dotted lines). Gravitational waves with \(\alphaM = -2\) remain stable but grow in amplitude for \(\alphaM < -2\).}


\section{Deviating propagation speed}

To explore the effect a modified propagating speed has on the evolution of gravitational waves in a modified gravity theory, I consider a parametric deviation \(\cT\) from the speed of light in the \namedeqref{parametrized evolution equation}{eq:evolution_friction}. The deviation of the propagation speed as it appears in \eqref{eq:evolution_friction} changes the effective wavelength scale \(\frac{1}{\cT k}\) such that the horizon entry of the gravitational wave occurs later for smaller values of \(\cT\). The horizon entry is discussed in detail in the context of general relativity in \autoref{sec:frw}.

\autoref{fig:varying_cT} shows the evolution of gravitational waves for several values of \(\cT=\const\). The plots were obtained numerically with the \mmainline{ParametricNDSolve} function in Mathematica and the same assumptions and initial conditions as chosen in \autoref{sec:param_friction_const}.

\plt{varying_cT}{Tensor perturbations \(\abs{h(a)}\) with a modified propagation speed \(\cT=\const\)}{Gravitational waves with lower propagation speed \(\cT\) exhibit a delayed horizon entry.}

Because gravitational waves with lower propagation speed \(\cT\) exhibit a delayed horizon entry, their amplitude today is closer to their initial value than for gravitational waves that propagate faster. An additional friction contribution \(\alphaM\) as discussed in \autoref{sec:param_friction_const} therefore is less effective for gravitational waves with lower \(\cT\) that enter the horizon later and thus have less time until today to decrease or increase in amplitude. This suggests a degeneracy between a modified propagation speed and an additional friction contribution, as both affect the amplitude of tensor perturbations measured today. A positive \(\alphaM\) that results in a faster decrease in amplitude can be compensated by a lower propagation speed \(\cT\) that in turn delays the horizon entry and thus gives the tensor mode less time to decrease. Also, the amount growing tensor modes increase in amplitude is reduced by a lower propagation speed.

\todo{is cT > 1 possible?}


\section{Late-time additional friction}

Many theories of modified gravity aim to solve the \emph{new cosmological constant problem} presented in \autoref{sec:cc_problem_new}. This requires such theories to model the accelerated expansion of our universe today as described in \autoref{sec:cc_intro}. Proposed modifications to general relativity thus generally only affect late-time cosmological regimes \todo{add reference/example, maybe specific f(R) theory} where the \LCDM standard model of cosmology relies on dark energy to accelerate the universe. Therefore, suitable parametrized modifications to the \namedeqref{evolution equation of gravitational waves}{eq:evolution_friction} are time-dependent and vanish in early-time cosmological regimes to reflect this behaviour of modified gravity theories.

In particular, I consider for the additional friction parametrized by \(\alphaM\) the time-dependent pa\-ra\-me\-tri\-za\-tion
\begin{equation}\label{eq:friction_param_powerlaw}
	\alphaM(t) = \alphaMnot \cdot a(t)^\betaexp \quad \text{with} \quad \alphaMnot = \const \quad \text{and} \quad \betaexp > 0
	\eqpunct{.}
\end{equation}
In an expanding universe where \(a(t) \in \sbr{0,1}\) monotonically increases with time, the additional friction contribution \(\alphaM\) vanishes at early times to recover general relativity, but deviates from \LCDM today.

\autoref{fig:varying_aM0_beta} shows numerical solutions to the \namedeqref{evolution equation}{eq:evolution_friction} with this particular parametrization for \(\alphaM\) for various values of both \(\alphaMnot\) and \(\betaexp\). It was obtained with the \mmainline{ParametricNDSolve} function in Mathematica where the same initial conditions and assumption as in \autoref{sec:param_friction_const} were chosen. The special case \(\betaexp=0\) corresponds to constant friction \(\alphaM(t)=\alphaMnot\) as investigated in \autoref{sec:param_friction_const}. Increasing \(\betaexp\) to positive values reduces the additional friction at early times where \(a(t) < 1\). Larger values for \(\betaexp\) amplify this effect. \todo{include \autoref{fig:varying_beta} at all?}

\plt{varying_aM0_beta}{Tensor perturbations \(\abs{h(a)}\) with parametrized additional friction \(\alphaM(t) = \alphaMnot \cdot a(t)^\betaexp\)}{For positive \(\betaexp\), the evolution matches \LCDM at early times and deviates at late times. Increasing \(\abs{\alphaMnot}\) yields larger deviations from \LCDM at late times.}

At late times, tensor modes will increase or decrease in amplitude with a slope approximating the behaviour of constant additional friction \(\alphaMnot\) discussed in \autoref{sec:param_friction_const}. However, tensor modes that grow in this regime according to \eqref{eq:grow_cond_const_fric} can be suppressed by a sufficiently large value of \(\betaexp\) remain physically viable. 


\chapter{Gravitational Waves in Parametrized Bigravity}

Bigravity is a class of modified gravity theories introduced in \autoref{sec:cc_problem_new} where a second \emph{reference metric} \(f\) in addition to the physical metric \(g\) is considered. Both metrics are coupled such that the evolution equation of gravitational waves in this bimetric setting becomes \citep{Amendola2015}
\begin{subequations}\label{eq:evolution_bimetric}
\begin{align}
	\ddconf{h}\ofmetr{n} + \del{2 + \alphaMofmetr{n}} \Hconf \dconf{h}\ofmetr{n} + \del{\Hconf^2 \mofmetr{n}^2 + \cTofmetr{n}^2 k^2} h\ofmetr{n} = \Hconf^2 \qofmetr{n} h\ofmetr{m} \quad \text{in conformal time \(\conft\)} \\
    \text{or} \quad \ddefold{h}\ofmetr{n} + \del{\frac{\defold{\Hconf}}{\Hconf} + 2 + \alphaMofmetr{n}} \defold{h}\ofmetr{n} + \del{\mofmetr{n}^2 + \frac{\cTofmetr{n}^2 k^2}{\Hconf^2}} h\ofmetr{n} = \qofmetr{n} h\ofmetr{m} \quad \text{in e-foldings \(\efold\)}
\end{align}
\end{subequations}
where the indices \(n,m \in \cbr{g,f}\) with \(n \neq m\) refer to the perturbations and parameters associated with the physical metric \(g\) and the reference metric \(f\), respectively. The additional friction terms~\(\alphaMofmetr{n}\) and the deviations from the speed of light~\(\cTofmetr{n}\) for each of the two metrics were discussed in \autoref{ch:param_mod_grav} in the context of unimetric modified gravity. Each metric also has an associated \emph{mass parameter}~\(\mofmetr{n}\) and a \emph{coupling}~\(\qofmetr{n}\) to the other metric. In the theory of bigravity, these parameters have specific, time-dependent forms.

\plt{bimetric}{}{Physical metric follows reference metric...}


\subsection{Comparison to \cite{Amendola2015}}

In \cite{Amendola2015}, both the physical and reference metric tensor perturbations can be described as \ref{eq:evolution_friction_regime} with friction
\begin{align}
	\alphaMofmetr{g} &= 0 \\
	\alphaMofmetr{f} &= -3(1+\eosp) < -2
\end{align}

Clearly, the physical metric perturbations $h_g$ fall like $\frac{1}{a}$ as in \LCDM, whereas the reference metric perturbations $h_f$ grow like $a^1$ in RDE and $a^\frac{1}{2}$ in MDE.



\chapter{Summary}


\begin{appendices}

\chapter{Mathematical Appendix}

\section{Derivation of the Einstein equations from the Einstein-Hilbert action}\label{app:deriv_einstein_eqns}

To derive the Einstein equations, we begin with the Einstein-Hilbert action

\begin{align}
	S[g] = \int \! \dif{^4x} \sqrt{-|g|} \Rscal \eqpunct{.}
\end{align}


\section{Derivation of the FRW metric from the cosmological principle}\label{app:deriv_frw}

\section{Time evolution of the scale factor in an FRW universe}\label{app:deriv_frw_a_evol}

\begin{align}
	0 &= \dt{\rho} + 3\Hcosm\del{\rho + p} = \dt{\rho} + \Hcosm\rho \nexp \quad \text{with} \quad \nexp \defeq 3\del{1 + \eosp} \\
	\implies &\diff{\rho}{\efold} = -\rho\nexp \implies \rho \propto a^{-\nexp} \\
	\implies &\text{for \(\spatcurv=0\):} \quad \dt{a} \propto a^{-\frac{\nexp}{2}+1} \implies a(t) \propto
	\begin{cases}
		t^{\frac{2}{\nexp}} &\text{for \(\eosp\neq-1\)} \\
		\eul^{\Hcosm t} &\text{for \(\eosp=-1\)}
	\end{cases}
\end{align}


\section{Decomposition of Metric Perturbations}\label{app:deriv_decompose}
\todo{show as decomposition of the SO(3) group}

\section{...}

\section{Bessel Functions}\label{app:bessel}

\begin{align}
	f''(x) + \left(2p+1\right)\frac{1}{x} f'(x) + \left(\alpha^2 + \frac{\beta^2}{x^2}\right) f(x) = 0 \\
	\implies f(x) = x^{-p} \left[C_1 J_q(\alpha x) + C_2 Y_q(\alpha x)\right] \quad \text{with} \quad q = \sqrt{p^2 - \beta^2}
\end{align}

\subsection*{Asymptotic Behaviour}
\begin{align}
	J_p(x) \, &\rightarrow \, \sqrt{\frac{2}{\pi x}} \cos(x-\frac{2p+1}{4}\pi) + \mathcal{O}(x^{-\frac{3}{2}}) \\
	Y_p(x) \, &\rightarrow \, \sqrt{\frac{2}{\pi x}} \sin(x-\frac{2p+1}{4}\pi) + \mathcal{O}(x^{-\frac{3}{2}})
\end{align}


\chapter{On the Anthropic Principle}\label{app:anthropic}

- cite Sheldon: The anthropic principle states that if we wish to explain why our universe exists the way it does, the answer is that it must have qualities that allow intelligent creatures to arise who are capable of asking the question. As I am doing so eloquently right now.

\end{appendices}

\plt{varying_beta}{}{For larger \(\betaexp\) the curve converges to \LCDM at early times and deviates at late times.}


\bibliography{lit,Bachelorarbeit}


\end{document}
