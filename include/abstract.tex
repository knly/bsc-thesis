\cleardoublepage
\thispagestyle{plain}

\makeatletter
\begin{center}
	\large\textbf{\@title}\\
	\normalsize\@author
\end{center}
\makeatother

\paragraph{Abstract}

Modified gravity theories generally aim to solve part of the \emph{cosmological constant problem} by providing self-accelerating cosmological solutions without a cosmological constant. Such modifications of general relativity also affect the evolution of gravitational waves in the proposed theory. Instead of focussing on an explicit model, I introduce parametric modifications to the evolution equation of gravitational waves in both unimetric and bimetric settings and investigate their effect on the evolution of tensor perturbation modes. In particular, I argue that any modified gravity theory that exhibits \emph{growing tensor modes} in cosmological evolution can be in tension with experiments. Therefore, parametric constraints for the physical viability of a general modified gravity theory can be found such that tensor modes remain within limits set by observations.

\begin{otherlanguage}{ngerman}

\paragraph{Zusammenfassung}

Viele Gravitationstheorien, die über die Allgemeine Relativitätstheorie hinausgehen, werden entwickelt um die beschleunigte Expansion unseres Universums ohne eine kosmologische Konstante zu erklären. Solche Veränderungen betreffen im Allgemeinen auch das Verhalten von Gravitationswellen in der vorgeschlagenen Theorie. Anstatt ein explizites Modell anzunehmen führe ich parametrische Anpassungen in der Gravitationswellengleichung ein und untersuche ihre Auswirkungen auf Tensorstörungen sowohl in unimetrischen als auch bimetrischen Gravitationstheorien. Insbesondere argumentiere ich, dass Modelle mit im Zeitverlauf des Universums anwachsenden Tensormoden in Konflikt mit experimentellen Beobachtungen stehen können. Somit können parametrische Einschränkungen für die physikalische Viabilität von Gravitationstheorien gefunden werden, sodass Tensormoden innerhalb der experimentellen Grenzen verbleiben.

\end{otherlanguage}
