\chapter{On the Anthropic Principle}\label{app:anthropic}

\epigraph{\itshape The anthropic principle states that if we wish to explain why our universe exists the way it does, the answer is that it must have qualities that allow intelligent creatures to arise who are capable of asking the question. As I am doing so eloquently right now.}{---Sheldon Cooper, \textit{The Big Bang Theory}\footnotemark}
\footnotetext{Quote from season 6, episode 1 of the CBS television series \textit{The Big Bang Theory}}

In our quest to understand nature, we postulate theories. A good theory, from a physicist's point of view, is based on a number of assumptions that give rise to many observable predictions. The more opportunities we have to falsify a theory, the more robust and trustworthy it is considered, given none of the tests require us to discard it. The most accepted theories of nature share \emph{first postulates}, such as the \emph{principle of least action}, that have withstood falsification so consistently, we believe they form fundamental principles of nature. Whether such principles can \emph{always} be traced back to an even more fundamental understanding of nature is subject of many discussions and a separate issue.

However, even our most successful physical theories, general relativity among them, include degrees of freedom that do not arise from first principles but must be determined experimentally. Particularly \emph{fundamental physical constants}, such as the fine-structure constant\footnote{The fine-structure constant, or Sommerfeld constant, determines the coupling strength of charged particles to the electromagnetic field in the standard model of particle physics.} and the cosmological constant, are considered free parameters that would drastically change the appearance of the universe to the point where observers like us were unable to exist, if they were not \emph{fine-tuned} to the particular value we observe. This problem is answered by employing a manifestation of the \emph{anthropic principle} that requires our universe to assume these values based on the concept that \emph{only a universe can be observed that allows observers to arise}.

Although the tautology in this statement is clear, it raises crucial questions about our ability to find fundamental laws of nature from our viewpoint within nature itself. However, the anthropic principle is easily abused and must be applied with care, since it inhibits any further scientific inquiries and provides no means of falsification for the theory when stated in the above sense. Historically, for instance, the \emph{flatness problem} of non-inflationary cosmology prompted many anthropic arguments until the theory of cosmological inflation revealed opportunities for a natural solution. An anthropic approach is still a viable theory, although difficult to test, granted that nature allows for a sufficient number of domains with varying conditions such that the required environmental selection can occur \autocite{Carroll2003}.

Instead of resorting to the anthropic principle in search for answers, fine-tuning problems present the opportunity to reconsider the most fundamental premises of our theory. Particularly those assumed implicitly, such as the classical notions of space and time, are challenging to identify but crucial to reach a more profound understanding of nature.
