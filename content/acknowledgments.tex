\chapter{Acknowledgements}

First and foremost, I would like to thank Dr. Valeria Pettorino and Prof. Dr. Luca Amendola for the opportunity to work in their research group and for the supervision of my bachelor project, as well as Dr. Matteo Martinelli for his invaluable support. My thanks also goes to the rest of the \emph{cosmici} group for the numerous and fruitful discussions about our universe and beyond. I must, however, apologize to anyone in the group who feels their personal opinion on the cosmological constant problem is not reflected explicitly enough in this thesis and suggest to reach a concensus over another barbecue, be it with lobsters or without.

I am also most grateful for the lecturers in theoretical physics that sparked my fascination in the fundamental theories of nature, both at the University of Heidelberg and at the \emph{Winter School on Gravity and Light 2015} in Linz, particularly Prof. Dr. Carlo Ewerz and Dr. Frederic Schuller for their excellent lectures on classical electrodynamics and general relativity, respectively. It was also in great parts Prof. Dr. Bernhard Schutz who awakened my interest in gravitational waves with his insightful lecture in Linz.

Furthermore, I wish to express my appreciation for the work done by the plethora of contributors to scientific toolkits such as the Python programming language, Mathematica and \LaTeX{} and their associated packages, since their importance in modern science is not acknowledged nearly enough as they deserve. Taken for granted in their daily use by many, these computational tools have had an unmeasurable impact on the advancement of science.

Finally, I want to thank everybody who spent their time in discussion with me, be it about physics or philosophy, or the fine line in between. I believe discussions have the unique nature that they provide insights one can not reach on one's own, and that any question that is voiced out of curiosity is worthy of being asked. Thus, I am grateful for the quality many physicists posses to reconsider even the most fundamental assumptions in our quest to understand the universe we live~in.
